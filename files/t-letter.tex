%D \module
%D   [       file=t-letter,
%D        version=2008.05.28,
%D          title=\CONTEXT\ User Module,
%D       subtitle=Framework for Letters,
%D         author=Wolfgang Schuster,
%D           date=\currentdate,
%D      copyright=Wolfgang Schuster]

%M \loadsetups[cont-en.xml]
%M \loadsetups[letter.xml]

%D \subject{Constants and Variables}

\writestatus{loading}{Context User Module / Framework for Letters}

\unprotect

%D I use a few extra constants and variables in my module.

\startconstants            all

                  graphic: graphic
                     head: head
                     foot: foot
              backaddress: backaddress
                  topmark: topmark
                  botmark: botmark
                  cutmark: cutmark
                  endmark: endmark
                 usermark: usermark
                 foldmark: foldmark
                extension: extension
                interface: interface
          backgroundimage: backgroundimage
               whitespace: whitespace
                 optimize: optimize
\stopconstants

\startvariables            all

                   letter: letter
               secondpage: secondpage
              backaddress: backaddress
                     foot: foot
                  topmark: topmark
                  botmark: botmark
                  cutmark: cutmark
                  endmark: endmark
                reference: reference
                 location: location
                   option: option
                  address: address
                 foldmark: foldmark
                 usermark: usermark
                  opening: opening
                  closing: closing
                    layer: layer
               letternext: letternext
               lettermain: lettermain
                  special: special
                 notation: notation
                   inside: inside
\stopvariables

%D Before we start the real module let us introduce two namespaces
%D for the style and content setup commands.

\def\????ld{@@@@ld} % LetterData
\def\????ls{@@@@ls} % LetterStyle
\def\????lv{@@@@lv} % LetterValue

\startmodule[letter]

\setupmodule
  [\c!style=,            % default letter style
   \c!extension=,        % default letter extension
   \c!interface=default] % default letter interface

%D \subject{Setup commands}
%D 
%D \macros{setupletterstyle}
%D 
%D The layout for the first page could be set with the \mono{firstpage}
%D key and the layout for the second and all following pages with the
%D \mono{secondpage} key. You could use any key and value from
%D \tex{setuplayout} to set your own letter layout.
%D 
%D \showsetup{setupletterstyle:layout}
%D 
%D The header and footer texts for the letter is set with the \mono{header}
%D and \mono{footer} keys, the text itself I written as argument as one of
%D the three keys. The command itself has no option to use different texts
%D for odd and even pages (if you really want this in a letter) but you could
%D solve this with the macro \tex{doifoddpageelse}, I will give you a short
%D example below.
%D 
%D \showsetup{setupletterstyle:text}
%D 
%D This setups could produce somtimes a unwanted result is you use a doublesided
%D layout, because the values for \mono{lefttext} and \mono{righttext} change on
%D odd and even pages their position, e.g. \mono{righttext} will always go to the
%D outer margin and \mono{lefttext} to the inner margin. Let's you want the
%D pagenumber always on the right side you could use the following code, you could
%D adapt it to your own needs without problems.
%D 
%D \starttyping
%D \setupletterstyle
%D   [header]
%D   [lefttext=\doifoddpageelse\donothing\pagenumber,
%D    righttext=\doifoddpageelse\pagenumber\donothing]
%D \stoptyping

\def\setupletterstyle
  {\dotripleempty\dosetupletterstyle}

\def\dosetupletterstyle[#1][#2][#3]%
  {\ifthirdargument
     \dodosetupletterstyle[#1][#2][#3]%
   \else\ifsecondargument
     \dodosetupletterstyle[#1][#2][]%
   \else
     \dodosetupletterstyle[\v!option][#1][]%
   \fi\fi}

\def\dodosetupletterstyle[#1][#2][#3]%
  {\processallactionsinset
     [#1]
     [ \v!firstpage=>{\definelayout[\v!letter\v!firstpage][#2]},
      \v!secondpage=>{\definelayout[\v!letter\v!secondpage][#2]},
          \v!header=>{\dosetupletterheader[#2][#3]},
          \v!footer=>{\dosetupletterfooter[#2][#3]},
          \v!option=>{\getparameters[\????ls\v!option][#2]},
        \v!foldmark=>{\dododosetupletterstyle[\v!topmark,\v!botmark,\v!cutmark][#2][#3]},
         \s!unknown=>{\dododosetupletterstyle[#1][#2][#3]}]}

%D Setup of the layers and their frames.

\def\dododosetupletterstyle[#1][#2][#3]%
  {\def\dodododosetupletterstyle##1%
     {\doifelsenothing{#3}
        {\ExpandSecondAfter\doifinsetelse{##1}\letter!list!sections
           {\dododododosetupletterstyle[##1][\v!option][#2]}
           {\ExpandSecondAfter\doifinsetelse{##1}\letter!list!descriptions
              {\setupdescriptions[\v!letter##1][#2]}
              {\dododododosetupletterstyle[##1][\v!layer,\v!frame,\v!option][#2]}}}
        {\dododododosetupletterstyle[##1][#2][#3]}}%
   \processcommacommand[#1]\dodododosetupletterstyle}

\def\dododododosetupletterstyle[#1][#2][#3]%
  {\def\dodododododosetupletterstyle##1%
     {\processaction
        [##1]
        [    \v!layer=>{\dododododosetupletterstyle[#1][\v!layer\v!a,\v!layer\v!b][#3]},
         \v!layer\v!a=>{\setuplayer[#1][#3]},
         \v!layer\v!b=>{\getparameters[\????ls#1\v!layer][#3]},
             \v!frame=>{\getparameters[\????ls#1\v!frame][#3]},
            \v!option=>{\getparameters[\????ls#1][#3]},
           \s!unknown=>{\getparameters[\????ls#1##1][#3]}]}%
   \processcommalist[#2]\dodododododosetupletterstyle}

%D Setup for the header and footer texts.

\newtoks\letter!toks!header
\newtoks\letter!toks!footer

\def\dosetupletterheader[#1][#2]%
  {\letter!toks!header\emptytoks
   \doifelsenothing{#2}
     {\letter!toks!header{\setupheader[\v!text][#1]}}
     {\letter!toks!header{\setupheader[#1][#2]}}}

\def\dosetupletterfooter[#1][#2]%
  {\letter!toks!footer\emptytoks
   \doifelsenothing{#1}
     {\letter!toks!footer{\setupfooter[\v!text][#1]}}
     {\letter!toks!footer{\setupfooter[#1][#2]}}}

%D \macros
%D   {setupletter}
%D
%D The \tex{setupletter} macro could be used from a interface, I define
%D only the macro and nothing else. You could also redefine the macro later
%D because it is not used from the raw interface.

%\def\setupletter
%  {\dodoubleempty\dosetupletter}
%
%\def\dosetupletter[#1][#2]%
%  {\ifsecondargument
%     \dodosetupletter[#1][#2]%
%   \else\iffirstargument
%     \dodosetupletter[\v!option][#1]%
%   \fi\fi}
%
%\def\dodosetupletter[#1][#2]%
%  {\def\dododosetupletter##1%
%     {\getparameters[\????ld##1][#2]}%
%   \processcommalist[#1]\dododosetupletter}

\def\setupletter
  {\dodoubleempty\dosetupletter}

\def\dosetupletter[#1][#2]%
  {\ifsecondargument
     \dodosetupletter[#1][#2]%
   \else
     \dodosetupletter[#1][]%
   \fi}

\def\dodosetupletter[#1][#2]%
  {\doifelsenothing{#2}
     {\getparameters[\????ld][#1]}
     {\def\dododosetupletter##1%
        {\getparameters[\????ld##1][#2]}%
      \processcommalist[#1]\dododosetupletter}}

\definecomplexorsimple\setlettervalue

\def\complexsetlettervalue[#1]#2%
  {\setvalue{\????ld#1}{#2}}

\def\simplesetlettervalue#1#2%
  {\setvalue{\????ld#1}{#2}}

%D A few extra macros to test for content of the letter values, don't rely
%D on them at the moment because they could change or I will remove, rename,
%D rewrite etc. them.

%\def\doiflettervalue#1#2%
%  {\doiftextelse{\lettervalue{#1}{#2}}
%     \firstofoneargument
%     \gobbleoneargument}
%
%\def\doifelselettervalue#1#2%
%  {\doiftextelse{\lettervalue{#1}{#2}}
%     \firstoftwoarguments
%     \secondoftwoarguments}

\def\doiflettervalue#1%
  {\doiftextelse{\lettervalue{#1}}
     \firstofoneargument
     \gobbleoneargument}

\def\doifelselettervalue#1%
  {\doiftextelse{\lettervalue{#1}}
     \firstoftwoarguments
     \secondoftwoarguments}

%D \subject{Interface, Style and Extension files}
%D 
%D \macros
%D   {useletterextension,useletterstyle,useletterinterface}
%D
%D This module provides different types of files beside the main module.
%D 
%D Two of them could be used by the user while the third is only used in
%D this file but you're free to use but you should really now what you do.
%D 
%D Each of the files has it's own file extension, you have to use if you
%D want to create such a file, this has the advantage to use the same name
%D for a collection of macros.

\def\letter!suffix!style    {nls}
\def\letter!suffix!extension{nle}
\def\letter!suffix!interface{nli}

%D Let us begin now to talk about the different types of files.
%D 
%D \subsubject{letter interface}
%D 
%D The letter interface provides the connection between the user and the
%D inner macros, it is possible to write a letter without a interface but
%D you will loose many of the nice features. The features itself are defined
%D within the interface and try to help the user to write a letter in
%D a simpler than the raw interface.
%D 
%D I want to show you the difference between the default interface and the
%D raw backend with a simple example, just a opening, the content and
%D a closing text.
%D 
%D The example with the default interface looks like:
%D 
%D \starttyping
%D \startletter
%D   [opening=Hello User,
%D    closing=Good luck with your own interface]
%D I will show you how you could write a letter with my letter module
%D and the difference between the default and the raw interface. The raw
%D interface is should be only used by a high level interface and not the
%D user but you're free to do it.
%D \stopletter
%D \stoptyping
%D 
%D The same example with the raw interface looks like:
%D 
%D \starttyping
%D \startsetups[letter:opening]
%D Hello User
%D \stopsetups
%D 
%D \startsetups[letter:content]
%D I will show you how you could write a letter with my letter module
%D and the difference between the default and the raw interface. The raw
%D interface is should be only used by a high level interface and not the
%D user but you're free to do it.
%D \stopsetups
%D 
%D \startsetups[letter:closing]
%D Good luck with your own interface
%D \stopsetups
%D \stoptyping
%D 
%D The interface itself is loaded with the \tex{useletterinterface}
%D command but you could also use the optional argument for \tex{usemodule}
%D to load it. The first is only used in this file but it makes sense
%D to use it also in a wrapper file for your own interface.
%D
%D \showsetup{useletterinterface}

\def\useletterinterface
  {\dosingleargument\douseletterinterface}

\def\douseletterinterface[#1]%
  {\def\dodouseletterinterface##1%
     {\readfile{##1.\letter!suffix!interface}\donothing\donothing}%
   \processcommalist[#1]\dodouseletterinterface}

%D \subsubject{letter extension}
%D 
%D The second filetype this module supports are letterextensions, you
%D could use it to save macros and other values you want to use in many
%D without the need to put it always in the real letter.
%D 
%D A few reasons for a letterextension file are:
%D 
%D \startitemize[packed,intro]
%D \item The letter head from your company,
%D \item your own macro for the reference line,
%D \item labels you need in different files (see \mono{label.nle}),
%D \item \unknown
%D \stopitemize
%D 
%D The syntax to load a letterextension file is the same as for a interface
%D and you could also argument for \tex{usemodule}.
%D 
%D \showsetup{useletterextension}

\def\useletterextension
  {\dosingleargument\douseletterextension}

\def\douseletterextension[#1]%
  {\def\dodouseletterextension##1%
     {\readfile{##1.\letter!suffix!extension}\donothing\donothing}%
   \processcommalist[#1]\dodouseletterextension}

%D \subsubject{letter styles}
%D 
%D After I mentioned above how to load a interface and extension
%D I will now come to the most important extension for the module.
%D 
%D Letter styles are used to change the layout of the letter,
%D set the values for layers and enable and disable common options.
%D Although it is possible to do all these things in the letter itself
%D it is better to save often used settings in a extra file you could
%D load on demand and share between your letters.
%D 
%D You could load a style file in the same way as the other to files,
%D the first is \tex{useletterstyle} and the second the optional argument
%D for \tex{usemodule}.
%D 
%D \showsetup{useletterstyle}

\def\useletterstyle
  {\dosingleargument\douseletterstyle}

\def\douseletterstyle[#1]%
  {\def\dodouseletterstyle##1%
     {\readfile{##1.\letter!suffix!style}\donothing\donothing}%
   \processcommalist[#1]\dodouseletterstyle}

%D \macros
%D   {letterstylevalue,lettervalue}
%D 
%D The two commands \tex{letterstylevalue} and \tex{lettervalue} could
%D be used to access the values for the keys used in the letterstyle
%D settings and the letter values for information.

%D The rest of module contains only internal macros and other settings,
%D you don't need them as a normal user but you look at the default.

\def\letterstylevalue#1#2%
  {\csname\????ls#1#2\endcsname}

%\def\lettervalue#1#2%
%  {\csname\????ld#1#2\endcsname}

\def\lettervalue#1%
  {\csname\????ld#1\endcsname}

\def\letter!list!marking
  {\v!topmark,\v!botmark,\v!cutmark,\v!endmark,\v!usermark}

\def\dodefineletterlayer[#1]%
  {\ExpandSecondAfter\doifelse{#1}\v!foldmark
     {\def\dodododefineletterlayer##1{\dododefineletterlayer[##1]}%
      \processcommacommand[\letter!list!marking]\dodododefineletterlayer}
     {\dododefineletterlayer[#1]}}

\def\dododefineletterlayer[#1]%
  {\setvalue{\????ls\v!option#1}{\v!yes}%
   \presetlocalframed[\????ls#1\v!frame]%
   \definelayer
     [#1]
     [\c!width=\paperwidth,
      \c!height=\paperheight]%
   \setupletterstyle
     [#1][\v!option]
     [\c!symbol=,
      \c!offset=\zeropoint,
      \c!leftmargin=\zeropoint,
      \c!rightmargin=\zeropoint]%
   \setupletterstyle
     [#1][\v!frame]
     [\c!frame=\v!off,
      \c!align=\v!right,
      \c!offset=\zeropoint,
      \c!strut=\v!yes]%
   \setupletterstyle
     [#1][\v!layer]
     [\c!state=\v!start,
      \c!offset=\zeropoint,
      \c!preset=\v!left\v!top]}

\def\dosetletterlayer[#1]%
  {\doifinsetelse{#1}{\v!topmark,\v!botmark,\v!cutmark,\v!endmark,\v!usermark}
     {\doifelsenothing{\letterstylevalue{#1}\c!symbol}
        {\dodosetletterlayer[#1][\setups{letter:#1}]}
        {\dodosetletterlayer[#1][\letterstylevalue{#1}\c!symbol]}}
     {\dodosetletterlayer[#1][\setups{letter:#1}]}}

\def\dodosetletterlayer[#1][#2]%
  {\setlayer[#1]
     {\localframed[\????ls#1\v!frame]
        {\doadaptleftskip {\letterstylevalue{#1}\c!leftmargin}%
         \doadaptrightskip{\letterstylevalue{#1}\c!rightmargin}%
         \doattributes{\????ls#1}\c!style\c!color{#2}}}}

\let\letter!list!sections\empty

\def\dodefinelettersection[#1]%
  {\setvalue{\????ls\v!option#1}{\v!yes}%
   \doifnotinset{#1}\letter!list!sections
     {\appendtocommalist{#1}\letter!list!sections}%
   \setupletterstyle
     [#1]
     [\v!option]
     [\c!before=,
      \c!after=,
      \c!align=,
      \c!leftmargin=\zeropoint,
      \c!rightmargin=\zeropoint,
      \c!style=,
      \c!color=]}

\def\dosetlettersection[#1]%
  {\begingroup
   \edef\@@lettersection{#1}%
   \doifelsenothing{\letterstylevalue\@@lettersection\c!before}
      \endgraf
     {\letterstylevalue\@@lettersection\c!before}%
   \doadaptleftskip {\letterstylevalue\@@lettersection\c!leftmargin}%
   \doadaptrightskip{\letterstylevalue\@@lettersection\c!rightmargin}%
   \doifsomething{\letterstylevalue\@@lettersection\c!align}
     {\setupalign[\letterstylevalue\@@lettersection\c!align]}%
   \dostartattributes{\????ls#1}\c!style\c!color\empty
     \setups[letter:#1]%
   \dostopattributes
   \doifelsenothing{\letterstylevalue\@@lettersection\c!after}
      \endgraf
     {\endgraf\letterstylevalue\@@lettersection\c!after}%
   \endgroup}

\let\letter!list!descriptions\empty

\def\dodefineletterdescription[#1]%
  {\setvalue{\????ls\v!option#1}{\v!yes}%
   \doifnotinset{#1}\letter!list!descriptions
     {\appendtocommalist{#1}\letter!list!descriptions}%
   \definedescription[\v!letter#1][\c!width=\zeropoint]}

\def\dosetletterdescription[#1]%
  {\csname\e!start\v!letter#1\endcsname
   \setups[letter:#1]%
   \csname\e!stop \v!letter#1\endcsname}

\def\doflushletterdescription
  {\def\dodoflushletterdescription##1%
     {\doif{\letterstylevalue\v!option{##1}}\v!yes{\dosetletterdescription[##1]}}%
   \processcommacommand[\letter!list!descriptions]\dodoflushletterdescription}

%D \subject{Placement of the content}
%D 
%D Enable the layouts for the first and second page.

\startsetups[letter:layout]

  \setuplayout[\v!letter\v!firstpage]

  \appendtoks\setuplayout[\v!letter\v!secondpage]\to\everyaftershipout

\stopsetups

%D Header for the first page, you could use it to place the logo from your
%D company or your own address and name.

\dodefineletterlayer[\v!head]

\startsetups[letter:place:head]

  \dosetletterlayer[\v!head]

\stopsetups

%D Address

\dodefineletterlayer[\v!address]

\startsetups[letter:place:address]

  \dosetletterlayer[\v!address]

\stopsetups

%D Could be used to place a backaddress above the normal address.

\dodefineletterlayer[\v!backaddress]

\startsetups[letter:place:backaddress]

  \dosetletterlayer[\v!backaddress]

\stopsetups

%D Reference line

\dodefineletterlayer[\v!reference]

\startsetups[letter:place:reference]

  \dosetletterlayer[\v!reference]

\stopsetups

%D Location field

\dodefineletterlayer[\v!location]

\startsetups[letter:place:location]

  \dosetletterlayer[\v!location]

\stopsetups

%D Footer block, you could use it to place information about your company
%D but take care set \mono{repeat=yes} if you want to place it on every page,
%D you should disable the footer for the second and following pages inb this
%D case.

\dodefineletterlayer[\v!foot]

\startsetups[letter:place:foot]

  \dosetletterlayer[\v!foot]

\stopsetups

%D Two additional layers from Hans Hagen's own letter module, it provide
%D for users who used \mono{m-letter.tex} and want to switch to my module.
%D 
%D I need they also for my pragma letter interface.

\dodefineletterlayer[\v!lettermain]

\startsetups[letter:place:lettermain]

  \dosetletterlayer[\v!lettermain]

\stopsetups

\dodefineletterlayer[\v!letternext]

\startsetups[letter:place:letternext]

  \dosetletterlayer[\v!letternext]

\stopsetups

%D Letterhead, american only

\dodefinelettersection[\v!letter\v!head]

\startsetups[letter:place:\v!letter\v!head]

  \dosetlettersection[\v!letter\v!head]

\stopsetups

%D Date Line, american only

\dodefinelettersection[\v!date\v!line]

\startsetups[letter:place:\v!date\v!line]

  \dosetlettersection[\v!date\v!line]

\stopsetups

%D Reference Line, american only

\dodefinelettersection[\v!reference\v!line]

\startsetups[letter:place:\v!reference\v!line]

  \dosetlettersection[\v!reference\v!line]

\stopsetups

%D Special Notation, american only

\dodefinelettersection[\v!special\v!notation]

\startsetups[letter:place:\v!special\v!notation]

  \dosetlettersection[\v!special\v!notation]

\stopsetups

%D Inside Address, american only

\dodefinelettersection[\v!inside\v!address]

\startsetups[letter:place:\v!inside\v!address]

  \dosetlettersection[\v!inside\v!address]

\stopsetups

%D Title

\dodefinelettersection[\v!title]

\startsetups[letter:place:\v!title]

  \dosetlettersection[\v!title]

\stopsetups

%D Subject

\dodefinelettersection[\v!subject]

\startsetups[letter:place:\v!subject]

  \dosetlettersection[\v!subject]

\stopsetups

%D Opening

\dodefinelettersection[\v!opening]

\startsetups[letter:place:\v!opening]

  \dosetlettersection[\v!opening]

\stopsetups

%D Letter content

\dodefinelettersection[\v!content]

\startsetups[letter:place:\v!content]

  \begingroup

  \doifsomething\@@@@lsoptionindenting{\setupindenting[\@@@@lsoptionindenting]}

  \dosetlettersection[\v!content]

  \endgroup

\stopsetups

%D Closing

\dodefinelettersection[\v!closing]

\startsetups[letter:place:\v!closing]

  \dosetlettersection[\v!closing]

\stopsetups

%D Apendices

\dodefinelettersection[\v!appendices]

\startsetups[letter:place:\v!appendices]

  \dosetlettersection[\v!appendices]

\stopsetups

%D Fold- and cutmarks, you enable and disable all of them independant
%D from any other or disable all marks with just one command.

\dodefineletterlayer[\v!foldmark]

\startsetups[letter:place:foldmark]

  % Upper foldmark

  \startlocalsetups[letter:place:topmark]

    \dosetletterlayer[\v!topmark]

  \stoplocalsetups

  % Lower foldmark

  \startlocalsetups[letter:place:botmark]

    \dosetletterlayer[\v!botmark]

  \stoplocalsetups

  % Cutmark

  \startlocalsetups[letter:place:cutmark]
	
    \dosetletterlayer[\v!cutmark]

  \stoplocalsetups

  % Endmark, a relict from the typewriter area

  \startlocalsetups[letter:place:endmark]

    \dosetletterlayer[\v!endmark]

  \stoplocalsetups

  % Free mark, you could use it to place your own symbols

  \startlocalsetups[letter:place:usermark]

    \dosetletterlayer[\v!usermark]

  \stoplocalsetups

  \doif\@@@@lsoptiontopmark \v!yes{\setups[letter:place:topmark]}
  \doif\@@@@lsoptionbotmark \v!yes{\setups[letter:place:botmark]}
  \doif\@@@@lsoptioncutmark \v!yes{\setups[letter:place:cutmark]}
  \doif\@@@@lsoptionendmark \v!yes{\setups[letter:place:endmark]}
  \doif\@@@@lsoptionusermark\v!yes{\setups[letter:place:usermark]}

\stopsetups

%D Background graphic:

\defineoverlay
  [\v!graphic] % choose a better name!
  [\doifsomething\@@@@lsoptionbackgroundimage
     {\overlayfigure{\@@@@lsoptionbackgroundimage}}]

\def\letter!list!backgrounds
  {\v!color,\v!letternext,\v!lettermain,\v!head,\v!foot,%
   \v!backaddress,\v!graphic,\v!address,\v!reference,\v!location,%
   \v!topmark,\v!botmark,\v!cutmark,\v!endmark,\v!usermark}

%D Settings at the begin of every letter.

\startsetups[letter:initialize]

  \@@@@lsoptionbefore

  \setupsubpagenumber
    [\c!number=\@@@@lsoptionpagenumber,
     \c!way=\v!by\v!text,
     \c!state=\@@@@lsoptionstate]

  \resetsubpagenumber

  \begingroup

  \page

  \setuppagenumbering
    [\c!alternative=\@@@@lsoptionalternative,
     \c!location=] % use the header and footer option instead

  \the\letter!toks!header
  \the\letter!toks!footer

  \setupbackgrounds
    [\v!page]
    [\c!background=\letter!list!backgrounds,
     \c!backgroundcolor=\@@@@lsoptionbackgroundcolor]

  \doifsomething\@@@@lsoptionbodyfont{\switchtobodyfont[\@@@@lsoptionbodyfont]}

  \doifsomething\@@@@lsoptionwhitespace{\setupwhitespace[\@@@@lsoptionwhitespace]}

\stopsetups

%D Settings at the end of every letter.

\startsetups[letter:finish]

  \page

  \setuplayout[\v!reset]

  \endgroup

  \resetsubpagenumber

  \@@@@lsoptionafter

\stopsetups

%D Place the letter.

\startsetups[letter:place]

  \setups[letter:initialize] % Settings at the begin
  \setups[letter:layout]     % Page layout
  \setups[letter:reset]      % Reset layers
  \setups[letter:layer]      % Place layers
  \setups[letter:sequence]   % Place content
  \setups[letter:finish]     % Settings at the end

\stopsetups

%D Reset the layer content.

\def\letter!list!layers
  {\v!topmark,\v!botmark,\v!cutmark,\v!endmark,\v!usermark,\v!letternext,\v!lettermain}

\def\dodoresetletterlayer#1%
  {\resetlayer[#1]}

\startsetups[letter:reset]

  \processcommacommand[\letter!list!layers]\dodoresetletterlayer
  
  \setuplayer[\v!letternext][\c!state=\v!continue]

\stopsetups

\startsetups[letter:layer]

  \doif\@@@@lsoptionmarking    \v!yes{\setups[letter:place:foldmark]}
  \doif\@@@@lsoptionhead       \v!yes{\setups[letter:place:head]}
  \doif\@@@@lsoptionbackaddress\v!yes{\setups[letter:place:backaddress]}
  \doif\@@@@lsoptionaddress    \v!yes{\setups[letter:place:address]}
  \doif\@@@@lsoptionreference  \v!yes{\setups[letter:place:reference]}
  \doif\@@@@lsoptionlocation   \v!yes{\setups[letter:place:location]}
  \doif\@@@@lsoptionfoot       \v!yes{\setups[letter:place:foot]}
  \doif\@@@@lsoptionlettermain \v!yes{\setups[letter:place:lettermain]}
  \doif\@@@@lsoptionletternext \v!yes{\setups[letter:place:letternext]}

\stopsetups

%D Place the letter content.

\let\letter!order!sections\letter!list!sections

\def\doflushlettersection#1%
  {\doif{\letterstylevalue\v!option{#1}}\v!yes{\setups[letter:place:#1]}}

\startsetups[letter:sequence]

  \executeifdefined{\v!letter\s!start\c!optimize\@@@@lsoptionoptimize}\donothing
  \processcommacommand[\letter!order!sections]\doflushlettersection
  \executeifdefined{\v!letter\s!stop \c!optimize\@@@@lsoptionoptimize}\donothing

\stopsetups

%D \subject{Default values}
%D 
%D Default settings for the option values of the letter.

\setupletterstyle
  [\v!option]
  [\c!marking=\v!yes,
   \c!endmark=\v!no,
   \c!usermark=\v!no,
   \c!backaddress=\v!no,
   \c!indenting=,
   \c!whitespace=,
   \c!backgroundcolor=\v!white,
   \c!backgroundimage=,
   \c!before=,
   \c!after=,
   \c!header=\v!reset,
   \c!footer=\v!reset,
   \c!pagenumber=1,
   \c!bodyfont=,
   \c!optimize=\v!no,
   \c!alternative=\v!singlesided,
   \c!state=\v!stop]

% interface always before styles and extension because users should
% start with a clean interface without predefined and unwanted values.

\useletterinterface[\currentmoduleparameter\c!interface]
\useletterstyle    [\currentmoduleparameter\c!style    ]
\useletterextension[\currentmoduleparameter\c!extension]

\stopmodule

\protect \endinput
