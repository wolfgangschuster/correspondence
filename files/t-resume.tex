%D \module
%D   [       file=t-resume,
%D        version=2008.11.27,
%D          title=\CONTEXT\ User Module,
%D       subtitle=Framework for Letters,
%D         author=Wolfgang Schuster,
%D           date=\currentdate,
%D      copyright=Wolfgang Schuster,
%D          email=schuster.wolfgang@googlemail.com,
%D        license=Public Domain]

\writestatus{loading}{Context User Module / Framework for Letters}

\unprotect

\usemodule[correspondence]

%D I use a few extra constants and variables in my module.

\startinterface all
  \setinterfacevariable {resume}     {resume}
\stopinterface

\def\????rd{@@@@rd} % ResumeData
\def\????rs{@@@@rs} % ResumeStyle

\startmodule[resume]

\setupmodule
  [\c!style=,
   \c!extension=,
   \c!interface=default]

%D \macros
%D   {setupresumestyle}

\definecorrespondencesetup[\v!resume\v!style][\v!resume][\????rs]

%D Setup for the header and footer texts.

\newtoks\resume!toks!header
\newtoks\resume!toks!footer

\def\dosetupresumeheader[#1][#2]%
  {\resume!toks!header\emptytoks
   \doifelsenothing{#2}
     {\resume!toks!header{\setupheader[\v!text][#1]}}
     {\resume!toks!header{\setupheader[#1][#2]}}}

\def\dosetupresumefooter[#1][#2]%
  {\resume!toks!footer\emptytoks
   \doifelsenothing{#1}
     {\resume!toks!footer{\setupfooter[\v!text][#1]}}
     {\resume!toks!footer{\setupfooter[#1][#2]}}}

%D \macros
%D   {setupresume}

\definecorrespondencevalue[\v!resume][\????ld]

%D A few extra macros to test for content of the letter values, don't rely
%D on them at the moment because they could change or I will remove, rename,
%D rewrite etc. them.
%D 
%D With the change of the \tex{setupletter} command it was also necessary
%D to change the definition for lettervalue tests.

\def\doifresumevalue         {\doifcorrespondencevalue         \????rd}
\def\doifelseresumevalue     {\doifelsecorrespondencevalue     \????rd}
\def\doifresumestylevalue    {\doifcorrespondencestylevalue    \????rs}
\def\doifelseresumestylevalue{\doifelsecorrespondencestylevalue\????rs}

%D \subject{Interface, Style and Extension files}
%D 
%D \macros
%D   {useletterextension,useletterstyle,useletterinterface}
%D
%D This module provides different types of files beside the main module.
%D 
%D Two of them could be used by the user while the third is only used in
%D this file but you're free to use but you should really now what you do.
%D 
%D Each of the files has it's own file extension, you have to use if you
%D want to create such a file, this has the advantage to use the same name
%D for a collection of macros.

\def\resume!suffix!style    {nrs}
\def\resume!suffix!extension{nre}
\def\resume!suffix!interface{nri}

\definecorrespondencefile[\v!resume\v!interface][\resume!suffix!interface]
\definecorrespondencefile[\v!resume\v!extension][\resume!suffix!extension]
\definecorrespondencefile[\v!resume\v!style    ][\resume!suffix!style    ]

%D \macros
%D   {letterstylevalue,lettervalue}

\def\resumestylevalue#1#2%
  {\csname\????rs#1#2\endcsname}

\def\resumevalue#1%
  {\csname\????rd#1\endcsname}

\def\resume!list!marking
  {\v!topmark,\v!botmark,\v!cutmark,\v!usermark}

\let\resume!list!layers      \empty
\let\resume!list!sections    \empty
\let\resume!list!descriptions\empty

\definecorrespondencelayer      [\v!resume][\????rs]
\definecorrespondencesection    [\v!resume][\????rs]
\definecorrespondencedescription[\v!resume][\????rs]

%D \subject{Placement of the content}
%D 
%D Enable the layouts for the first and second page.

\startsetups[\v!resume:\v!layout]

  \setuplayout[\v!resume\v!firstpage]

  \appendtoks\setuplayout[\v!resume\v!secondpage]\to\everyaftershipout

\stopsetups

%D Header for the first page, you could use it to place the logo from your
%D company or your own address and name.

\dodefineresumelayer[\v!head]

\startsetups[\v!resume:\v!place:\v!head]

  \dosetresumelayer[\v!head]

\stopsetups

%D Footer block, you could use it to place information about your company
%D but take care set \mono{repeat=yes} if you want to place it on every page,
%D you should disable the footer for the second and following pages inb this
%D case.

\dodefineresumelayer[\v!foot]

\startsetups[\v!resume:\v!place:\v!foot]

  \dosetresumelayer[\v!foot]

\stopsetups

%D Layer to set head and foot blocks at the second and all following pages.

\dodefineresumelayer[\v!nexthead]

\startsetups[\v!resume:\v!place:\v!nexthead]

  \dosetresumelayer[\v!nexthead]

\stopsetups

\dodefineresumelayer[\v!nextfoot]

\startsetups[\v!resume:\v!place:\v!nextfoot]

  \dosetresumelayer[\v!nextfoot]

\stopsetups

%D Title

\dodefineresumesection[\v!title]

\startsetups[\v!resume:\v!place:\v!title]

  \dosetresumesection[\v!title]

\stopsetups

%D Letter content

\dodefineresumesection[\v!content]

\startsetups[\v!resume:\v!place:\v!content]

  \begingroup

  \doifresumestylevalue\v!option\c!indenting{\setupindenting[\@@@@rsoptionindenting]}

  \dosetresumesection[\v!content]

  \endgroup

\stopsetups

%D Apendices

\dodefineresumesection[\v!appendices]

\startsetups[\v!resume:\v!place:\v!appendices]

  \dosetresumesection[\v!appendices]

\stopsetups

%D Fold- and cutmarks, you enable and disable all of them independant
%D from any other or disable all marks with just one command.

\dodefineresumelayer[\v!foldmark]

\startsetups[\v!resume:\v!place:\v!foldmark]

  % Upper foldmark

  \startlocalsetups[\v!resume:\v!place:\v!topmark]

    \dosetresumelayer[\v!topmark]

  \stoplocalsetups

  % Lower foldmark

  \startlocalsetups[\v!resume:\v!place:\v!botmark]

    \dosetresumelayer[\v!botmark]

  \stoplocalsetups

  % Cutmark

  \startlocalsetups[\v!resume:\v!place:\v!cutmark]
	
    \dosetresumelayer[\v!cutmark]

  \stoplocalsetups

  % Free mark, you could use it to place your own symbols

  \startlocalsetups[\v!resume:\v!place:\v!usermark]

    \dosetresumelayer[\v!usermark]

  \stoplocalsetups

  \doif\@@@@rsoptiontopmark \v!yes{\setups[\v!resume:\v!place:\v!topmark]}
  \doif\@@@@rsoptionbotmark \v!yes{\setups[\v!resume:\v!place:\v!botmark]}
  \doif\@@@@rsoptioncutmark \v!yes{\setups[\v!resume:\v!place:\v!cutmark]}
  \doif\@@@@rsoptionusermark\v!yes{\setups[\v!resume:\v!place:\v!usermark]}

\stopsetups

%D Background graphic:

\defineoverlay
  [\v!resume:\c!backgroundimage]
  [\doifsomething\@@@@rsoptionbackgroundimage
     {\overlayfigure{\@@@@rsoptionbackgroundimage}}]

\defineoverlay
  [\v!resume:\c!background]
  [\@@@@rsoptionbackground]

\def\resume!list!backgrounds
  {\v!resume:\c!backgroundimage,\v!resume:\c!background,%
   \v!resume:\v!head,\v!resume:\v!foot,\v!resume:\v!nexthead,%
   \v!resume:\v!nextfoot,\v!resume:\v!topmark,\v!resume:\v!botmark,%
   \v!resume:\v!cutmark,\v!resume:\v!usermark}

%D Settings at the begin of every letter.

\startsetups[\v!resume:\v!initialize]

  \@@@@rsoptionbefore

  \setupsubpagenumber
    [\c!number=\@@@@rsoptionpagenumber,
     \c!way=\v!by\v!text,
     \c!state=\@@@@rsoptionstate]

  \resetsubpagenumber

  \begingroup

  \page

  \setuppagenumbering
    [\c!alternative=\@@@@rsoptionalternative,
     \c!location=] % use the header and footer option instead

  \the\resume!toks!header
  \the\resume!toks!footer

  \setupbackgrounds
    [\v!paper]
    [\c!background=\v!color,
     \c!backgroundcolor=\@@@@rsoptionbackgroundcolor]

  \setupbackgrounds
    [\v!page]
    [\c!background=\resume!list!backgrounds,
     \c!backgroundcolor=\@@@@rsoptionbackgroundcolor]

  \doifresumestylevalue\v!option\c!bodyfont  {\switchtobodyfont[\@@@@rsoptionbodyfont]}

  \doifresumestylevalue\v!option\c!whitespace{\setupwhitespace[\@@@@rsoptionwhitespace]}

\stopsetups

%D Settings at the end of every letter.

\startsetups[\v!resume:\v!finish]

  \page

  \setuplayout[\v!reset]

  \endgroup

  \resetsubpagenumber

  \@@@@rsoptionafter

\stopsetups

%D Place the letter.

\startsetups[\v!resume:\v!place]

  \setups[\v!resume:\v!initialize] % Settings at the begin
  \setups[\v!resume:\v!layout]     % Page layout
  \setups[\v!resume:\v!reset]      % Reset layers
  \setups[\v!resume:\v!layer]      % Place layers
  \setups[\v!resume:\v!sequence]   % Place content
  \setups[\v!resume:\v!finish]     % Settings at the end

\stopsetups

%D Reset the layer content.

\def\resume!list!resetlayers
  {\v!topmark,\v!botmark,\v!cutmark,\v!usermark}

\def\dodoresetresumelayer#1%
  {\resetlayer[\v!resume:#1]}

\startsetups[\v!resume:\v!reset]

  \processcommacommand[\resume!list!resetlayers]\dodoresetresumelayer

\stopsetups

\startsetups[\v!resume:\v!layer]

  \doif\@@@@rsoptionmarking \v!yes{\setups[\v!resume:\v!place:\v!foldmark]}
  \doif\@@@@rsoptionhead    \v!yes{\setups[\v!resume:\v!place:\v!head    ]}
  \doif\@@@@rsoptionfoot    \v!yes{\setups[\v!resume:\v!place:\v!foot    ]}
  \doif\@@@@rsoptionnexthead\v!yes{\setups[\v!resume:\v!place:\v!nexthead]}
  \doif\@@@@rsoptionnextfoot\v!yes{\setups[\v!resume:\v!place:\v!nextfoot]}

\stopsetups

%D Place the letter content.

\let\resume!order!sections\resume!list!sections

\def\doflushresumesection#1%
  {\doif{\resumestylevalue\v!option{#1}}\v!yes{\setups[\v!resume:\v!place:#1]}}

\startsetups[\v!resume:\v!sequence]

  \processcommacommand[\resume!list!sections]\doflushresumesection

\stopsetups

%D \subject{Default values}

\setupresumestyle
  [\v!option]
  [\c!marking=\v!yes,
   \c!usermark=\v!no,
   \c!indenting=,
   \c!whitespace=,
   \c!background=,
   \c!backgroundcolor=\v!white,
   \c!backgroundimage=,
   \c!before=,
   \c!after=,
   \c!header=\v!reset,
   \c!footer=\v!reset,
   \c!pagenumber=1,
   \c!bodyfont=,
   \c!alternative=\v!singlesided,
   \c!state=\v!stop]

%D Default style

\useresumeinterface[\currentmoduleparameter\c!interface]
\useresumestyle    [\currentmoduleparameter\c!style    ]
\useresumeextension[\currentmoduleparameter\c!extension]

\stopmodule

\protect \endinput
