\startcomponent lm-resume-interface

\environment lm-environment

\chapter{Interface}

\page

\section{Default}

\setup[resume]

\page

\section{moderncv}

The moderncv interface is based on the layout of the moderncv-package for \LaTeX{}
and with the same elements and styles. The difference between the \LaTeX{} and the
\ConTeXt-version is that you don't need a \type{\maketitle} command to place the
header for each style and it is placed by default.

\startframedtext[width=\textwidth,framecolor=red,align=middle,corner=00]
The current implementation of the moderncv interface is written as resume style and
not as interface, this means you load it with \type{\useresumestyle} or as argument
for \type{\usemodule} with the style key.
\stopframedtext

The available elements for the interface are described below.

\subsubject{\tex{cvline}}

The \type{\cvline} take two arguments, the first place the text in the left margin
and accepts only short entries, the second argument accepts longer texts which are
placed in the text area. You could change the distance after the entry with the
optional argument. 

\startnointerference
\setup[cvline]
\stopnointerference

\starttyping
\cvline{...}{...}
\stoptyping

The two lines

\starttyping
\cvline{supervisors}{Supervisors}
\cvline{description}{\tx Short thesis abstract}
\stoptyping

results in

\start

\useresumestyle[modern]

\startelement
\cvline{supervisors}{Supervisors}
\cvline{description}{\tx Short thesis abstract}
\stopelement

\stop

\subsubject{\tex{cvlistitem}}

The \type{\cvlistitem} is similar to the \type{\cvline} command above but it
takes only one argument which is placed in the text area. In the left margin
appear a symbol like a itemize, you could change this symbol with the optional
argument before the text.

\startnointerference
\setup[cvlistitem]
\stopnointerference

\starttyping
\cvlistitem[<symbol>]{...}
\stoptyping

The two example line below

\starttyping
\cvlistitem{Item 1}
\cvlistitem[+]{Item 2}
\stoptyping

result in the following output

\start

\useresumestyle[modern]

\startelement
\cvlistitem{Item 1}
\cvlistitem[+]{Item 2}
\stopelement

\stop

\subsubject{\tex{cvlistdoubleitem}}

The \type{\cvlistdoubleitem} is a enhanced version of the \type{\cvlistitem} command
and place two text side by side, if you change the symbol it affects the symbol for
both texts. 

\startnointerference
\setup[cvlistdoubleitem]
\stopnointerference

\starttyping
\cvlistdoubleitem[<symbol>]{...}{...}
\stoptyping

As you can see in the next example you have to write the text in two command if you
want one below the other.

\starttyping
\cvlistdoubleitem[$\circ$]{Item 1}{Item 3}
\cvlistdoubleitem[$\circ$]{Item 2}{Item 4}
\stoptyping

You can see this in the output from the example above.

\start

\useresumestyle[modern]

\startelement
\cvlistdoubleitem[$\circ$]{Item 1}{Item 3}
\cvlistdoubleitem[$\circ$]{Item 2}{Item 4}
\stopelement

\stop

\subsubject{\tex{cventry}}

The \type{\cventry} has six argument and prints the argument two to five in the first
line of the text with a certain format for each text and separates them with commas.
The first argument is typed in the left margin and the last in the text on a separate
line.

\startnointerference
\setup[cventry]
\stopnointerference

\starttyping
\cventry{...}{...}{...}{...}{...}{...}
\stoptyping

Two possible settings for the arguments are:

\starttyping
\cventry{year--year}{Degree}{Institution}{City}{\it Grade}{Description}
\cventry{year--year}{Job title}{Employer}{City}{}{Description}
\stoptyping

This results in the time period in the left margin and the information in the text.

\start

\useresumestyle[modern]

\startelement
\cventry{year--year}{Degree}{Institution}{City}{\it Grade}{Description}
\cventry{year--year}{Job title}{Employer}{City}{}{Description}
\stopelement

\stop

\subsubject{\tex{cvlanguage}}

The \type{\cvlanguage} behaves also line the \type{\cvline} command but take
a third argument which can be used for additional information in the right margin.

\startnointerference
\setup[cvlanguage]
\stopnointerference

\starttyping
\cvlanguage{...}{...}{...}
\stoptyping

The following example

\starttyping
\cvlanguage{language 1}{Skill level}{Comment}
\cvlanguage{language 2}{Skill level}{Comment}
\stoptyping

results in:

\start

\useresumestyle[modern]

\startelement
\cvlanguage{language 1}{Skill level}{Comment}
\cvlanguage{language 2}{Skill level}{Comment}
\stopelement

\stop

\subsubject{\tex{cvcomputer}}

The \type{\cvcomputer} command is a enhanced version of the \type{\cvlistdoubleitem}
with two explicit argument for the label text which is written on the left of each entry.

\startnointerference
\setup[cvcomputer]
\stopnointerference

\starttyping
\cvcomputer{...}{...}{...}{...}
\stoptyping

As you can see in the first and third argument a label is written to describe text
category of each entry.

\starttyping
\cvcomputer{category 1}{XXX, YYY, ZZZ}{category 3}{XXX, YYY, ZZZ}
\cvcomputer{category 2}{XXX, YYY, ZZZ}{category 4}{XXX, YYY, ZZZ}
\stoptyping

This produce the following output.

\start

\useresumestyle[modern]

\startelement
\cvcomputer{category 1}{XXX, YYY, ZZZ}{category 3}{XXX, YYY, ZZZ}
\cvcomputer{category 2}{XXX, YYY, ZZZ}{category 4}{XXX, YYY, ZZZ}
\stopelement

\stop

\stopcomponent
